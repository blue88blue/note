\documentclass[UTF8]{ctexart} %使用ctex包,中文支持
\usepackage{amsmath}  %数学公式
\usepackage{graphicx} %插图
\usepackage{fancyhdr} %个性化页眉页脚
\usepackage{geometry} %页边距
\usepackage{bm}  % 公式加粗
\usepackage{float} %为了在分栏下插入图片
\usepackage{ulem}  % 换行下划线
%\usepackage{setspace} %行间距
\usepackage{multicol} %用于实现在同一页中实现不同的分栏
\geometry{a4paper,left=2cm,right=2cm,top=2cm,bottom=2cm} % 页边距设置

\title{信息论笔记}
\author{宋佳欢}
\pagestyle{plain}

\begin{document}
	\maketitle
	\tableofcontents
	\songti \zihao{-4}
	
	\section{信息熵}
		信息量$I(x)=f\big(p(x)\big)$,函数$f$需满足下列四个条件:
		
		\quad1.$f$单调递减,事件发生的概率越小,获得的信息量越大。
		
		\quad2.当$p(x)=1$,$f\big(p(x)\big)=0$
		
		\quad3.当$p(x)=0$,$f\big(p(x)\big)=\infty$
		
		\quad4.两件独立事件同时发生的获取的信息之和为$I(x,y)=I(x)+I(y)=f\big(p(x)\big)+f\big(p(y)\big)=f\big(p(x,y)\big)$
		
		因此,$p(x,y)=p(x)p(y)$。根据这个关系,$I(x)$与$p(x)$一定为对数关系。
		
		根据上述四个条件可得:\[I(x)=-logp(x)\]
		其中负号是用来保证信息量是正数或者零。而$log$函数基的选择是任意的(信息论中基常常选择为2,因此信息的单位为比特bits,即信息需要的编码长度;而机器学习中基常常选择为自然常数,因此单位常常被称为奈特nats;底数为10,单位则为Hart)。
		
		$I(x)$也被称为随机变量x的自信息 (self-information),\uline{描述的是随机变量的某个事件发生所带来的信息量}。
		
		现在假设一个发送者想传送一个随机变量的值给接收者。那么在这个过程中,他们传输的平均信息量可以通过求$I(x)$关于概率分布$p(x)$的期望求得,随机变量X的\uline{信息熵}的定义:
		\[H(X)= -\sum_{i=1}^np(x_i)logp(x_i)\]
		熵越大,随机变量的不确定性就越大。是对所有可能发生的事件产生的信息量的期望。
		
		\subsection{代数性质}
			\subsubsection{对称性}
				变量$p_1,p_2,\cdots,p_r$的顺序任意互换,熵不变。
				\[H(p_1,p_2,\cdots,p_r) = H(p_2,\cdots,p_r,p_1) = H(p_r,p_1,p_2,\cdots,p_{r-1})\]
			\subsubsection{非负性}
				\[H(p_1,p_2,\cdots,p_r) \geq 0\]
			\subsubsection{连续性}
				$H(p_1,p_2,\cdots,p_r)$ 是$p_i$的连续函数。
			\subsubsection{扩展性}
				\[\lim_{\varepsilon\rightarrow0}H(p_1,p_2,\cdots,p_i-\varepsilon,\cdots,p_r,\varepsilon_1,\varepsilon_2,\cdots,\varepsilon_k)=H(p_1,p_2,\cdots,p_r)\]
				其中每个$\varepsilon$都趋于0。当信源的消息集中的消息数增多时,因为这些消息对于的概率很小(比重很小),所以信源的熵不变。
			\subsubsection{可加性}
				统计独立的两个信源X,Y的两个联合信源的熵等于分别熵之和:
				\[H(X,Y) = H(X)+H(Y)\]	
				\[\begin{aligned}
					H(X,Y)&=-\sum_{i=1}^n\sum_{j=1}^mp_iq_jlogp_iq_j\\
					& =-\sum_{i=1}^n\sum_{j=1}^mp_iq_jlogp_i-\sum_{i=1}^n\sum_{j=1}^mp_iq_jlogq_j\\
					& =-\sum_{i=1}^n\big(\sum_{j=1}^mp_iq_j\big)logp_i-\sum_{j=1}^m\big(\sum_{i=1}^np_iq_j\big)logq_j\\
					& =-\sum_{i=1}^np_ilogp_i-\sum_{j=1}^mq_jlogq_j = H(X)+H(Y)
				\end{aligned}\]
			\subsubsection{递增性}
				将信源X中的其中一个元素划分成m个元素,这m个元素的概率之和等于原元素的概率,熵增加。
				\[H_{n+m-1}(p_1,p_2,\cdots,p_{n-1},q_1,q_2,\cdots,q_m) = H_n((p_1,p_2,\cdots,p_{n-1},p_n)
				+p_nH_m(\frac{q_1}{p_n},\frac{q_2}{p_n},\cdots,\frac{q_m}{p_n})\]
		\subsection{解析性质}
			\subsubsection{最大离散熵定理}
				在离散信源情况下,信源各符号等概率分布时,熵达到最大。(概率分布越接近平均分布,熵越大)
				\[H(p_1,p_2,\cdots, p_r) \leq H(\frac{1}{r},\frac{1}{r},\cdots,\frac{1}{r})\leq logr\]
			\subsubsection{上凸性}
				熵函数是严格的上凸函数:
				\[H(\theta P_1+(1-\theta) P_2) > \theta H( P_1)+(1-\theta)H(P_2)\]
	\section{信道}
		离散单符号信道可用传递概率表示:
			\[ 
			\begin{bmatrix} 
			P(b_1|a_1) &  P(b_2|a_1) & \cdots & P(b_s|a_1)\\
			P(b_1|a_2) &  P(b_2|a_2) & \cdots & P(b_s|a_2)\\
			\vdots & \vdots &   & \vdots\\
			(b_1|a_r) &  P(b_2|a_r) & \cdots & P(b_s|a_r)
			\end{bmatrix} 
			 \]
			 a为输入,b为输出。且传递矩阵(信道矩阵)每一行的元素相加等于1,即:
			 \[\sum_{j=1}^sP(b_j|a_i)=1\]
			 $P(b_i|a_i)$表示发送a收到b的概率(前向概率)描述了信道噪声的特征,$P(a_i|b_i)$表示接受到了$b_i$,发送端发送$a_i$的概率(后向概率)。
		\subsection{互信息与平均互信息}
			互信息(Mutual Information):$I(a_i;b_j)$表示接受到$b_j$后,能从$b_j$获得的关于$a_i$的信息量。
			互信息的三种写法:
			\[\begin{aligned}
				I(a_i;b_j) &= I(a_i) - I(a_i | b_j)\\
				& =I(b_j) - I(b_j | a_i)\\
				&  = I(a_i) + I(b_j) -I(a_i,b_j)
			\end{aligned}\]
			
			但是单个样本的互信息不足以表示整个系统,因此需要对多个样本取期望,即平均互信息:
			\[I(X;Y) = E_{P(X,Y)}\{I(a_i;b_j)\}\]
			
			对于单个样本的互信息,其值可正可负或为零,但是平均互信息一定不会为负值。
			
			证明:
			\[I(X;Y) = E_{P(X,Y)}\{I(a_i;b_j)\} = \sum_i\sum_j P(a_i,b_j)log\frac{P(a_i,b_j)}{P(a_i)P(b_j)}\]
			\[-I(X;Y) = \sum_i\sum_j P(a_i,b_j)log\frac{P(a_i)P(b_j)}{P(a_i,b_j)}\]
			根据琴生不等式,以及:
			\[\sum_i\sum_jP(a_i)P(b_j)=1\]
			所以
			 \[-I(X;Y) = \sum_i\sum_j P(a_i,b_j)log\frac{P(a_i)P(b_j)}{P(a_i,b_j)} \leq log\Big(\sum_i\sum_j P(a_i,b_j)\frac{P(a_i)P(b_j)}{P(a_i,b_j)}\Big)=log1=0\]
			 即
			 \[I(X;Y) \geq0\]
			 
			 互信息有三种写法,平均信息也衍生出三种写法:
				\[\begin{aligned}
				I(X;Y) &= H(X) - H(X|Y)\\
				& =H(Y)-H(Y|X)\\
				&  =H(X)+H(Y)-H(XY)
				\end{aligned}\]
			 其中$H(X|Y)$称为疑义度,$H(Y|X)$称为噪声熵,$H(XY)$称为联合熵(共熵)。
			 
			
	
		
\end{document}