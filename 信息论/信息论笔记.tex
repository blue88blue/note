\documentclass[UTF8]{ctexart} %使用ctex包,中文支持
\usepackage{amsmath}  %数学公式
\usepackage{graphicx} %插图
\usepackage{fancyhdr} %个性化页眉页脚
\usepackage{geometry} %页边距
\usepackage{bm}  % 公式加粗
\usepackage{float} %为了在分栏下插入图片
\usepackage{ulem}  % 换行下划线
%\usepackage{setspace} %行间距
\usepackage{multicol} %用于实现在同一页中实现不同的分栏
\geometry{a4paper,left=2cm,right=2cm,top=2cm,bottom=2cm} % 页边距设置

\title{信息论笔记}
\author{宋佳欢}
\pagestyle{plain}

\begin{document}
	\maketitle
	\tableofcontents
	\songti \zihao{-4}
	
	\section{信息的度量}
		信息量$I(x)=f\big(p(x)\big)$,函数$f$需满足下列四个条件:
		
		\quad1.$f$单调递减,事件发生的概率越小,获得的信息量越大。
		
		\quad2.当$p(x)=1$,$f\big(p(x)\big)=0$
		
		\quad3.当$p(x)=0$,$f\big(p(x)\big)=\infty$
		
		\quad4.两件独立事件同时发生的获取的信息之和为$I(x,y)=I(x)+I(y)=f\big(p(x)\big)+f\big(p(y)\big)=f\big(p(x,y)\big)$
		
		因此,$p(x,y)=p(x)p(y)$。根据这个关系,$I(x)$与$p(x)$一定为对数关系。
		
		根据上述四个条件可得:\[I(x)=-logp(x)\]
		其中负号是用来保证信息量是正数或者零。而$log$函数基的选择是任意的(信息论中基常常选择为2,因此信息的单位为比特bits,即信息需要的编码长度;而机器学习中基常常选择为自然常数,因此单位常常被称为奈特nats;底数为10,单位则为Hart)。
		
		$I(x)$也被称为随机变量x的自信息 (self-information),\uline{描述的是随机变量的某个事件发生所带来的信息量}。
		
		现在假设一个发送者想传送一个随机变量的值给接收者。那么在这个过程中,他们传输的平均信息量可以通过求$I(x)$关于概率分布$p(x)$的期望求得,随机变量X的\uline{信息熵}的定义:
		\[H(X)= -\sum_{i=1}^np(x_i)logp(x_i)\]
		熵越大,随机变量的不确定性就越大。是对所有可能发生的事件产生的信息量的期望。

		`
	
	
		
\end{document}