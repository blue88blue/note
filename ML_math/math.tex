\documentclass[UTF8]{ctexart} %使用ctex包,中文支持
\usepackage{amsmath}  %数学公式
\usepackage{graphicx} %插图
\usepackage{fancyhdr} %个性化页眉页脚
\usepackage{geometry} %页边距
%\usepackage{setspace} %行间距
\usepackage{multicol} %用于实现在同一页中实现不同的分栏
\geometry{a4paper,left=2cm,right=2cm,top=2cm,bottom=2cm} % 页边距设置

\title{机器学习数学笔记}
\author{宋佳欢}

\pagestyle{plain}

\begin{document}
	\maketitle
	\tableofcontents
	\songti \zihao{-4}
	
	\section{线性代数}
		\subsection{向量}
		(1)向量相加减:
		\[\begin{pmatrix}a\\b\\c\end{pmatrix}  \pm \begin{pmatrix}d\\e\\f\end{pmatrix}
		=\begin{pmatrix}a\pm d\\b\pm e\\c\pm f\end{pmatrix}\]
		
		(2)向量内积:\[\begin{pmatrix}a&b&c\end{pmatrix} \begin{pmatrix}d\\e\\f\end{pmatrix}
		=a\times d+b\times e+c\times f\]
		
		(3)向量的范数:\[\|x\|_p=\Big(\sum_{i=1}^n|x_i|^p\Big)^\frac{1}{p}\]
		
		\quad \quad L1范数:  \[\|x\|_1=\sum_{i=1}^n|x_i|\]
		
		\quad \quad L2范数(向量的模):  \[\|x\|_2=\sqrt{\sum_{i=1}^n(x_i)^2}\]
		
		\subsection{矩阵}
		(1)矩阵乘法:
		\[\begin{cases}
			(AB)C=A(BC)\\
			(A+B)C=AC+BC\\
			A(B+C)=AB+AC\\
			(AB)^T=B^TA^T
		\end{cases}\]
		
		(2)逆矩阵:
		\[\begin{cases}
		(AB)^{-1}=B^{-1}A^{-1}\\
		\big(A^{-1}\big)^{-1}=A\\
		\big(A^T\big)^{-1}=\big(A^{-1}\big)^T
		\end{cases}\]
		
		(3)梯度:
		\[\nabla f(c)=\big(\frac{\partial f}{\partial x_1},\cdots,\frac{\partial f}{\partial x_n}\big)^T\]
		
		(4)雅克比矩阵:		
		\[\begin{pmatrix}
		\frac{\partial y_1}{\partial x_1} & \frac{\partial y_1}{\partial x_2} &\cdots & \frac{\partial y_1}{\partial x_n}\\
		\frac{\partial y_2}{\partial x_1} & \frac{\partial y_2}{\partial x_2} &\cdots & \frac{\partial y_2}{\partial x_n}\\
		\cdots & \cdots & \cdots & \cdots\\
		\frac{\partial y_m}{\partial x_1} & \frac{\partial y_m}{\partial x_2} &\cdots & \frac{\partial y_m}{\partial x_n}
		\end{pmatrix}\]
		
		多输入多输出函数$y=f(x)$,其中y与x分别为m,n维向量。y的每个分量分别关于x的每个分量求偏导。
		
		(5)Hessian矩阵:
		\[\begin{pmatrix}
		\frac{\partial f}{\partial x_1^2} & \frac{\partial f}{\partial x_1\partial x_2} &\cdots & \frac{\partial f}{\partial x_1\partial x_n}\\
		\frac{\partial f}{\partial x_1\partial x_2} & \frac{\partial f}{\partial x_2^2} &\cdots & \frac{\partial f}{\partial x_2\partial x_n}\\\
		\cdots & \cdots & \cdots & \cdots\\
		\frac{\partial f}{\partial x_n\partial x_2} & \frac{\partial f}{\partial x_n\partial x_2} &\cdots & \frac{\partial f}{\partial x_n^2}\\
		\end{pmatrix}\]
		
		Hessian矩阵可判断多元函数的凹凸性,若Hessian矩阵正定,则函数是凸的,若负定则是凹的。
		
		(6)正定矩阵:$x^TAx>0,x!=0$,则矩阵A为正定矩阵。
		
		\qquad 判断条件:
		
		\qquad \qquad 矩阵的特征值全大于0
		
		\qquad \qquad 矩阵的所有顺序主子式都大于0
		
		\qquad \qquad 矩阵合同于单位阵
		
		(7)二次型:$x^TAx>0$
		
		\subsection{特征值与特征向量}
		对于n阶方阵A,有$Ax=\lambda x$,x为特征向量,x非零,$\lambda$ 为特征值。
		
		(1)特征值与特征向量的求解:
		\[(A-\lambda I)x=0\]
		
		根据方程组解的理论,齐次方程组有解,则$(A-\lambda I)$要为不满秩矩阵(即方程数量小于未知数数量,方程组有无数个解)。可得$|A-\lambda I|=0$,解出特征值。
		
		(2)特征值分解:正交矩阵P,$P^{-1}=P^T$,使得
		\[P^{-1}AP=\Lambda\]
		
		$\Lambda$为对角矩阵
		
		\subsection{矩阵和向量的求导}
		(1)
		\[\nabla W^Tx=W\]
		
		
		证明:
		
		\qquad 根据向量内积公式,$W^Tx=\sum_{i=1}^nw_ix_i$。
		
		\qquad 且对于x中任意的变量想$x_j$求偏导:
		\[\frac{\partial {\sum_{i=1}^nw_ix_i}}{\partial x_j}=w_j\]
		
		\qquad	可得:$\nabla W^Tx=\begin{pmatrix}
		w_1\\w_2\\\vdots\\w_n
		\end{pmatrix}=W$
		
		(2)
		\[
		\nabla x^TAx=(A+A^T)x
		\]
		
		证明:
		\[\nabla x^TAx = \sum_{i-1}^n\sum_{j=1}^na_{ij}x_ix_j\]
		
		\qquad 关于$x_k$求偏导,可能存在两种情况,即$i=k,j=k$,当i=k时有:
		
		\[ \frac{\partial{\sum_{j=1}^na_{kj}x_kx_j}}{\partial x_k} = 
		\sum_{j=1}^n a_{kj} x_j\]
		
		\qquad 相当于A的某一行与x的内积,当$k\subset [1,n]$, 相当于Ax;
		同理当j=k时,相当于A的某一列的转置与x内积,即$A^Tx$。二者相加 ,得证。
		
		(3)
		\[\nabla^2 x^TAx=A+A^T\]
		
		可由(1)(2)式证明。
		
		\section{概率论}
		
		\subsection{条件概率}
		\[p(b|a) = \frac{p(a,b)}{p(a)}\]
		\[p(a|b) = \frac{p(a,b)}{p(b)}\]
		
		\qquad $p(a,b)$表示a,b都发生的概率。$p(b|a)$表示a发生的情况下,b发生的概率。
		
		\subsection{贝叶斯公式}
		\[p(a|b)= \frac{p(a)p(b|a)}{p(b)}\]
		
		\qquad 证明:
		\[p(a|b)p(b)=p(a)p(b|a)=p(a,b)\]
		
		\qquad a为因,b为果,因此$p(b|a)$称为先验概率,$p(a|b)$称为后验概率。
		
		\subsection{随机事件的独立性}
		\[p(a,b)=p(a)p(b)\]
		
		\qquad 两个事件独立,表明二者发生的概率没有关联。即:
		\[\frac{p(a,b)}{p(a)}= p(b)=p(b|a)\]
		
		\subsection{期望与方差}
		(1)期望
		
		\qquad 对于离散随机变量:
		\[E(x)=\sum x_i p(x_i)\]
		
		\qquad 对于连续随机变量($f(x)$为概率密度函数):
		\[E(x)=\int_{-\infty}^\infty xf(x)dx\]	
		
		(2)方差(反应数据的波动程度)
			
		\qquad 离散:
		\[D(x)=\sum(x_i-E(x))^2 p(x_i)\]
		
		\qquad 连续:
		\[D(x)=\int_{-\infty}^\infty (x-E(x))^2f(x)dx\]
		
		\qquad 其他计算方式:$D(x)=E(x^2)-(E(x))^2$
		
		\subsection{常用分布}
		(1)正态分布(均值$\mu$,方差$\sigma^2$):
		\[f(x)=\frac{1}{\sqrt{2\pi}\sigma} e^{-\frac{(x-\mu)^2}{2\sigma^2}}\]
		
		(2)均匀分布:
		\[f(x)=\begin{cases}
			\frac{1}{b-a} \quad a\leq x \leq b\\
			0\quad x<a,x>b
		\end{cases}\]
		
		(3)二项分布:
		\[p(x=1)=p,\quad p(x=0)=1-p\]
		
		\subsection{随机向量}
		将变量x扩展成多维向量,如2维$x=[x_1,x_2]$
		\[f(x_1,x_2)\geq0\]
		\[\int_{-\infty}^\infty dx_1 dx_2=1\]
		
		\subsection{协方差}
		协方差反应的是两个随机变量之间线性相关的程度。定义为:
		\[\begin{aligned}
			cov(x_1,x_2)= E\Big(\big(x_1-E(x_1)\big)\big(x_2-E(x_2)\big)\Big)\\
			=E(x_1x_2)-E(x_1)E(x_2)
		\end{aligned}\]
		
		协方差矩阵(对为n维随机向量),对称的
		\[\begin{pmatrix}
			cov(x_1,x_1)&cov(x_1,x_2)&\cdots&cov(x_1,x_n)\\
			cov(x_2,x_1)&cov(x_2,x_2)&\cdots&cov(x_2,x_n)\\
			\cdots&\cdots&\cdots&\cdots\\
			cov(x_n,x_1)&cov(x_n,x_2)&\cdots&cov(x_n,x_n)\\
		\end{pmatrix}\]
		
		\subsection{最大似然估计}
		概率分布$p(x|\theta)$,从该分布中采样$x_i,i=1,\cdots,l$,各个样本之间互相独立。任务:在已知样本的情况下推测产生这些数据的模型参数$\theta$。因为采样事件已经完成,我们有理由相信这些样本一定程度上反应了实际的分布,寻找一个最符合当前观测样本的概率分布。整个采样事件的概率为所有样本概率之积(似然函数):
		\[L(\theta)=\prod_{i=1}^l p(x_i|\theta)\]
		
		最大化整个采样事件的可能性,但是概率乘积趋近于0,且不便于求导,两边取对数转换成连加函数(对数似然函数):
		\[lnL(\theta) = \sum_{i=1}^l lnp(x_i|\theta)\]
		\[max \sum_{i=1}^l lnp(x_i|\theta)\]
		
		
		

	
		
		
	
	
\end{document}